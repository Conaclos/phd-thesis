\chapter{Notations mathématiques}
\label{ch:math-notations}

\minitoc
\bigskip

Ce chapitre groupe l'ensemble des notations mathématiques utilisées dans ce manuscrit.

\section{Notations}

$f: I \to O$ représente une fonction totale dont le domaine de définition de $f$ est $\trm{dom}(f) = I$.
$m: K \pto V$ représente une fonction partielle ou une fonction totale où $\trm{dom}(m) \subseteq K$.
Il peut également représenter un dictionnaire qui associe des clefs d'un ensemble $K$ à des valeurs d'un ensemble $V$.
Une fonction $m$ peut être représentée comme un ensemble de couples ordonnés $\set*{\tuple*{k, m(k)} \given k \in \trm{dom}(m)}$.
Deux fonctions représentées par les mêmes ensembles sont égales.
Nous utilisons des types dépendants pour rendre totale la majorité des fonctions partielles que nous pouvons rencontrer.
Par exemple, la fonction qui soustrait un entier naturel $b$ à un entier naturel $a$, et qui donne un entier naturel est une fonction partielle.
En effet la fonction n'est pas définit lorsque $b$ est strictement supérieur à $a$.
cette fonction partielle $sub: \mathbb{N}_0 \times \mathbb{N}_0 \pto \mathbb{N}_0$ peut être exprimer comme une fonction totale en utilisant des types dépendants~: $sub: \set*{a \in \mathbb{N}}_0 \times \set*{b \in \mathbb{N}_0 \given b \leq a} \to \mathbb{N}_0$.

$\powerfset{E}$ représente l'ensemble des sous-ensembles finis de $E$, i.e. $\set*{X \subseteq E \given X\ \textit{is finite}}$.
$\card{E}$ représente la cardinalité de l'ensemble $E$.
Nous utilisons la notation $\forall T : \trm{Toy} \qsep \tuple*{T, \trm{play}}$ pour exprimer le fait que tout ensemble $T$, qui appartient à la famille d'ensembles $\trm{Toy}$, est équipé d'une fonction $\trm{play}$.
Afin de lever de potentiels ambiguïtés, nous pouvons indexer de telles fonctions avec l'ensemble correspondant ($\trm{play}_U$ for $U : \trm{Toy}$).
Pour éviter toute ambiguïté, nous notons $\mathbb{N}_0$ l'ensemble des entiers naturels qui inclut $0$, et $\mathbb{N}^*$ l'ensemble des entiers naturels qui n'inclut pas $0$.


