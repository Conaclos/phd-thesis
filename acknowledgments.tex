\begin{ThesisAcknowledgments}
Avant d'être une épreuve intellectuelle, le doctorat est d'abord une épreuve émotionnelle.
Cet aspect est souvent sous-estimé ou invisibilisé.
Les études sur le sujet existent et mettent en exergue le mal-être qui touche les doctorants et les doctorantes de toutes disciplines.
Profondément attaché à l'amélioration de la société, je fais le vœu que le doctorat soit à l'avenir plus sain ou, à défaut, cède sa place à d'autres formes d'aventures intellectuelles et humaines.

Ceci dit, mon doctorat a été une aventure intellectuelle stimulante dans laquelle j'ai pu nourrir ma curiosité.
Cette aventure a été l'occasion de nouer des liens avec mes encadrants François Charoy et Gérald Oster.
Je les remercie de m'avoir témoigné leur confiance et de m'avoir donné carte blanche pour mener à bien cette thèse.
Je remercie également Claudia-Lavinia Ignat, chargée de recherche à Inria, pour les échanges constructifs que nous avons eus.

Que serait ma thèse sans l'amitié et la bienveillance de mes camarades de travail~?
Je pense en particulier à Matthieu Nicolas avec qui j'ai eu des échanges critiques et qui a été d'un soutien infaillible.
Je pense à Quentin Laporte Chabasse avec qui j'ai eu des débats passionnés.
Je pense à Phillippe Kalitine, à Linda Ouchaou, et à Anis Ahmed Nacer qui ont chaque jour apporté leur niaque.
Je pense à Hoai-Le Nguyen qui a partagé avec modestie sa passion de la raquette.
Je pense également à Abir Ismaili-Alaoui, Alexandre Bourbeillon, Amina Abdmeziem, Béatrice Linot, Cédric Enclos, Chahrazed Labba, Clélie Amiot, Guillaume Rosinosky, Hoang Long Nguyen, Jean-Philippe Eisenbarth, Pierre-Antoine Rault, Riyadh Abdmeziem, et Siavash Atarodi.

De nombreuses personnes ont contribué indirectement à cette thèse et de manière très diverse.
Je pense aux employés du restaurant d'Inria Grand Est qui apportent chaque jour leur énergie pour faire sourire et proposer des repas de qualité.
Je pense aux employés administratifs qui m'ont permis de mener à bien les procédures qui jonchent le parcours du doctorat.
Je pense au personnel d'entretien des bâtiments qui préservent un cadre de travail propre et fonctionnel.

Je souhaiterais remercier mes ami·e·s de longue date Alexandre Merlin, Alizée Dulliand, et Rado Randrianomanana pour avoir été à mes côtés durant ces années de thèse.
Je remercie également ma sœur Alice Elvinger, mes parents Clarisse Elvinger et Jean-Michel Elvinger, mes grands-parents maternels Marie-Thérèse Mollet et Michel Mollet, mes grands-parents paternels Jeanine Elvinger et Claude Elvinger, ainsi que mes grands-parents de cœur Bernadette Petitcolas et René Petitcolas.
Ils et elles ont tous et toutes contribué à ce que je suis aujourd'hui.
\end{ThesisAcknowledgments}

\begin{ThesisDedication}
    Je dédie cette thèse à toute personne désireuse\\
    d'un monde plus libre et plus juste.
\end{ThesisDedication}
