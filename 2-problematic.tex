
\chapter{Problématiques}\label{ch:problematic}

\minitoc{}
\bigskip

% Spécifier la convergence
% deux pairs qui observe les mêmes effets d'opérations convergent
% Ici on parle d'op locales, elles peuvent être synchronisées de différentes
% manières.
% L'idée est de se centrer sur la confluence et de se déplacer au niveau de
% la couche réseau.

%La production collective de contenus par plusieurs centaines, voire plusieurs milliers d'utilisateur·ice·s est aujourd'hui une réalité.
%Par exemple, au sein des réseaux sociaux, des milliers d'utilisateur·ice·s réagissent en direct à des évènements et co-organisent des évènements.
%Cependant, la diffusion des activités massives de collaboration reste encore restreinte.
%Cette diffusion se heurte aux limitations conceptuelles des applications et des outils actuels de collaboration.
%Ces limitations sont induites par les infrastructures de collaboration sur lesquelles ils reposent.

Les infrastructures pair-à-pair de collaboration peuvent supporter les activités massives de collaboration.
Elles tolèrent les latences, les pannes, ainsi que les partitions réseaux.
Les collaborateur·ice·s peuvent modifier le contenu partagé de manière isolée.
Des collaborateur·ice·s géographiquement éloigné·e·s ou avec des connexions réseaux lentes et peu fiables peuvent contribuer à une collaboration.
%Ces infrastructures sont donc bien adaptées à des collaboration qui impliquent des appareils mobiles.
%Pour autant, elles assurent des latences basses lorsque les connexions réseaux le permettent.
Les collaborateur·ice·s ont donc l'illusion d'observer les modifications en temps réel lorsqu'elles et ils modifient simultanément un contenu.
Ces caractéristiques permettent de passer à l'échelle et de respecter le rythme de vie, ainsi que la manière de collaborer de chaque collaborateur·ice.

Pour ce faire, les infrastructures pair-à-pair de collaboration reposent sur la \emph{réplication de données}~\autocite{saito_2005_optimisticreplication}.
Chaque collaborateur·ice est associé·e à un pair qui possède une copie du contenu partagé.
% Un pair interroge et modifie sa copie indépendamment des autres pairs.
% En d'autres termes, les pairs n'ont pas besoin de se coordonner.
% La copie est donc \emph{toujours disponible}~\autocite{mahajan_2011_cac} pour être interrogée ou modifiée. % par son/sa collaborateur·ice
% C'est ce qui permet de tolérer les latences et les partitions réseaux.
% La modification parallèle des copies conduit inévitablement à leur divergence~: les copies fournissent des réponses distinctes à une ou plusieurs interrogations qui peuvent leur être soumis.
% Les pairs échangent à terme leurs modifications.
Les algorithmes de réplication de données sont responsables de la convergence des copies.
% Ils doivent s'assurer qu'à terme les copies répondent de manière identique à toutes les interrogations qui peuvent leur être soumit.
La convergence des copies conditionne le succès d'une collaboration.
Si les collaborateur·ice·s ne sont pas capables de partager une vue globalement convergente dans des délais raisonnables, leur collaboration est probablement compromise.
Assurer à terme la convergence des copies est donc important.

Les algorithmes existants de réplication de données supposent généralement l'absence de collaborateur·ice·s malintentionné·e·s.
Plus largement, ils supposent l'absence d'un adversaire actif.
Cette hypothèse n'est pas réaliste au sein d'infrastructures pair-à-pair.
Les pairs partagent les mêmes responsabilités et peuvent nuire au succès de la collaboration.
Certains pairs peuvent par exemple mener des attaques pour assurer une divergence permanente des copies des autres pairs.
Dans ce manuscrit de thèse, nous nous intéressons à cette problématique et nous proposons des mécanismes de défense pour assurer à terme la convergence.

Ce chapitre introduit dans un premier temps la réplication de données. Il développe ensuite la notion de convergence.
Il termine sur le modèle du système auquel nous nous intéressons et il définit l'adversaire du système.

\section{Réplication de données}\label{sec:optimistic-replication}
\index{replication-donnees@Réplication de données}

La réplication de données maintient des copies de contenus partagés sur un ensemble de pairs.
Elle augmente la disponibilité d'un contenu en permettant son interrogation et sa modification en dépit de la défaillance de pairs.
Les répliques sont souvent géographiquement réparties.
Ce qui rapproche les données des utilisateur·ice·s.
Elle réduit donc les latences pour interroger les contenus partagés.

Comparé à un contenu non-répliqué, un contenu répliqué a un nombre étendu de comportements.
Par exemple, des collaborateur·ice·s distinct·e·s peuvent modifier des copies distinctes d'un contenu répliqué.
Ces nouveaux comportements peuvent engendrer des scénarios difficiles à prévoir et traiter.
Idéalement, l'interrogation et la modification d'un contenu répliqué devrait se comporter comme s'il existait une seule copie.
Dans ce cas, nous disons que le contenu partagé est \emph{fortement cohérent} et nous parlons de réplication pessimiste de données~\autocite{saito_2005_optimisticreplication}.
La \autoref{fig:strong-consistency} met en évidence que l'interrogation d'un contenu fortement cohérent prend toujours en compte toutes les modifications précédemment effectuées.

\begin{figure}[htb]
\newcommand*\hsep{2.2}
\newcommand*\vsep{-1.4}
\centering
\begin{subfigure}{\linewidth}
    \centering
    \begin{tikzpicture}
        % Peers
        \node(A) at (0,0) {$p_A$};
        \node(B) at (0,\vsep) {$p_B$};
        \node (C) at (0,2*\vsep) {$p_C$};
        % Events
        \path (A)
            to +(\hsep,0) node[
                label=above:{$\trm{add}(1)$}
            ] (a2) {$\bullet$}
            to +(5*\hsep,0) node (aend) {}
        ;
        \path (B)
            to +(2*\hsep,0) node[
                label=above:{$\trm{rd}\set*{1}$}
            ] (b2){$\bullet$}
            to +(4*\hsep,0) node[
                label=above:{$\trm{rd}\set*{1,2}$}
            ] (b3){$\bullet$}
            to +(5*\hsep,0) node (bend) {}
        ;
        \path (C)
            to +(3*\hsep,0) node[
                label=above:{$\trm{add}(2)$}
            ] (c2) {$\bullet$}
            to +(5*\hsep,0) node (cend) {}
        ;
        % Timeline
        \foreach \src/\dest in {A/a2,B/b2,b2/b3,C/c2,a2/aend,b3/bend,c2/cend}
            \draw[timeline] (\src) to (\dest);
    \end{tikzpicture}
    \caption{}\label{subfig:strong-consistency-trace}
\end{subfigure}
\par\medskip
\begin{subfigure}{\linewidth}
    \centering
    \begin{tikzpicture}
        % Peers
        %\node (space) at (0,1.6) {};
        \node (V) at (0,0) {$p_V$};
        % Events
        \path (V)
            to +(\hsep,0) node[
                label=above:{$\trm{add}(1)$}
            ] (v2) {$\bullet$}
            to +(2*\hsep,0) node[
                label=above:{$\trm{rd}\set*{1}$}
            ] (v3){$\bullet {\scriptstyle}$}
            to +(3*\hsep,0) node[
                label=above:{$\trm{add}(2)$}
            ] (v4) {$\bullet {\scriptstyle}$}
            to +(4*\hsep,0) node[
                label=above:{$\trm{rd}\set*{1,2}$}
            ] (v5){$\bullet {\scriptstyle}$}
            to +(5*\hsep,0) node (vend) {}
        ;
        % Timeline
        \foreach \src/\dest in {V/v2,v2/v3,v3/v4,v4/v5,v5/vend}
            \draw[timeline] (\src) to (\dest);
    \end{tikzpicture}
    \caption{}\label{subfig:strong-consistency-single-copy}
\end{subfigure}
%
\caption[Réplication pessimiste d'un ensemble]{\subref{subfig:strong-consistency-trace} Réplication pessimiste d'un ensemble.
L'ensemble répliqué supporte une opération de modification $\set*{\trm{add}(n) \given n \in \mathbb{N}_0}$ qui ajoute un entier dans l'ensemble, et une opération d'interrogation $\trm{rd}$ qui donne la composition de l'ensemble.
Ainsi l'exécution de l'opération $\trm{add}(1)$ ajoute l'entier $1$ à l'ensemble, et nous notons $\trm{rd} \set*{1}$ une opération de lecture suivie de la valeur retournée par son exécution.
L'ensemble est initialement vide.
Le pair $p_A$ ajoute l'entier $1$.
Ultérieurement le pair $p_C$ ajoute l'entier $2$.
Le pair $p_B$ interroge à deux reprises l'ensemble.
Sa première interrogation se déroule entre les deux ajouts.
La seconde interrogation se déroule après les deux ajouts.
L'ensemble est fortement cohérent, une modification qui survient avant une autre modification ou une interrogation est donc visible à cette dernière.
La première interrogation retourne un singleton composé de l'élément $1$ et la seconde interrogation retourne un ensemble composé des éléments $1$ et $2$.
La cohérence forte maintient l'illusion de l'interrogation et la modification d'une seule copie.
\subref{subfig:strong-consistency-single-copy} L'histoire de l'exécution de cette copie abstraite peut être obtenue en projetant toutes les opérations sur un pair abstrait $p_V$.
}\label{fig:strong-consistency}
\end{figure}

Le maintien de l'illusion de l'interrogation et de la modification d'une seule copie a un coût.
Elle requiert une coordination étroite entre les pairs\index{coordination@Coordination}.
Une coordination qui peut se traduire par des communications systématiques avant chaque modification, voire avant chaque interrogation.
Les latences et les défaillances inhérentes aux réseaux~\cite{rotem_falalcies_2006} rendent cette coordination problématique.
Elle engendre plusieurs problèmes~:

\begin{description}
    \item[Disponibilité.] Le contenu ne peut plus être modifié, voire interrogé, lorsque des pairs sont isolés par des partitions réseaux.
    \item[Latence.] L'interrogation et la modification du contenu partagé est sujet aux latences du réseau et des pairs.
    La lenteur d'un pair ou d'une connexion réseau peut donc affecter l'ensemble du système.
    \item[Passage à l'échelle.] L'ajout de pairs, et donc de copies, peut détériorer la disponibilité et les latences, étant donné que le nombre de pairs impliqués dans une coordination augmente.
\end{description}

Le théorème CAP~\cite{brewer_cap_2000,gilbert_cap_2002} montre l'impossibilité de garantir à la fois une cohérence forte et une disponibilité à tout moment en présence de partitions réseaux.
Nous pouvons illustrer cette impossibilité en prenant l'exemple d'une bibliothèque municipale qui possède un système répliqué d'emprunts de livres.
Deux usagers souhaitent faire un emprunt.
Les usagers sont connecté·e·s à des pairs distincts.
Supposons que le réseau est momentanément partitionné de manière à ce que les pairs considérés se trouvent dans l'impossibilité de communiquer.
Le système doit faire le choix entre le maintien d'une cohérence forte des emprunts ou la disponibilité d'emprunter.
Le maintien de la cohérence forte empêche tout emprunt tant que les partitions réseaux persistent.
La disponibilité d'emprunter en dépit du partitionnement peut conduire à des conflits.
Par exemple, des usagers peuvent effectuer des emprunts sur la même période d'un livre disponible en un seul exemplaire.

L'illustration proposée pourrait nous amener à penser que la cohérence forte peut être difficilement sacrifiée au bénéfice de la disponibilité.
En effet, le risque d'emprunts conflictuels pourrait produire des situations désagréables pour les usagers et la bibliothèque.
L'illustration proposée souffre du problème de la \emph{double-dépense}~\cite{chohan_doublespending_2017}.
L'emprunt d'un exemplaire d'un livre sur une période donnée correspond à une ressource consommable.
Cette ressource ne peut pas être consommée (dépensée) deux fois.
Ce problème n'est pas toujours rencontré.
Par exemple, un système de pétition en ligne ne souffre pas du problème de la \emph{double-dépense}.
Plusieurs personnes peuvent signer une pétition sans avoir besoin de se coordonner.

Les applications n'ont pas toujours besoin de maintenir une cohérence forte.
En général, le maintien d'une cohérence plus faible est suffisant~\autocite{terry_baseball_2013, hellerstein_calm_2019, decandia_dynamo_2007}.
Prenant cette observation en considération, la \textbf{réplication optimiste}~\autocite{saito_2005_optimisticreplication} sacrifie la cohérence forte pour atteindre une haute disponibilité et des latences basses.
Elle autorise la modification et l'interrogation des copies sans aucune coordination.
Les pairs modifient indépendamment leur copie.
Ce qui conduit à la divergence des copies.
Deux copies divergentes fournissent des réponses différentes à au moins une interrogation.
Les pairs se synchronisent pour échanger leurs modifications.
Ils intègrent les modifications de chacun à leur copie.
Les modifications sont donc intégrées dans des ordres distincts.
Les protocoles de réplication optimiste sont responsables de la convergence des copies.
La \autoref{fig:optimistic-replication} illustre un scénario de réplication optimiste.

\begin{figure}[htb]
\newcommand*\hsep{2.2}
\newcommand*\vsep{-1.4}
\centering
\begin{tikzpicture}
    % Peers
    \node(A) at (0,0) {$p_A$};
    \node (B) at (0,\vsep) {$p_B$};
    % Event streams
    \path (A)
        to +(\hsep,0) node[
            label=above:{$\trm{add}(1)$}
        ] (a2) {$\bullet$}
        to +(2*\hsep,0) node[
            label=above:{$\trm{rd}\set*{1}$}
        ] (a3) {$\bullet$}
        to +(3*\hsep,0) node (a4) {$\bullet$}
        to +(4*\hsep,0) node (a5) {$\bullet$}
        to +(5*\hsep,0) node[
            label=above:{$\trm{rd}\set*{1,2}$}
        ] (a6) {$\bullet$}
        to +(6*\hsep,0) node (aend) {}
    ;
    \path (B)
        to +(\hsep,0) node[
            label=below:{$\trm{add}(2)$}
        ] (b2) {$\bullet$}
        to +(2*\hsep,0) node[
            label=below:{$\trm{rd}\set*{2}$}
        ] (b3) {$\bullet$}
        to +(3*\hsep,0) node (b4) {$\bullet$}
        to +(4*\hsep,0) node (b5) {$\bullet$}
        to +(5*\hsep,0) node[
            label=below:{$\trm{rd}\set*{1,2}$}
        ] (b6){$\bullet$}
        to +(6*\hsep,0) node (bend) {}
    ;
    % hb
    \foreach \src/\dest in {A/a2,a2/a3,a3/a4,a4/a5,a5/a6,B/b2,b2/b3,b3/b4,b4/b5,b5/b6,a6/aend,b6/bend}
        \draw[timeline] (\src) to (\dest);
    \draw[hb] (b4) to (a5);
    \draw[hb] (a4) to node[fill=white]{\footnotesize sync} (b5);
\end{tikzpicture}
\caption[Réplication optimiste d'un ensemble]{Exemple de réplication optimiste d'un ensemble.
Les pairs $p_A$ et $p_B$ débutent avec un ensemble vide.
$p_A$ ajoute l'élément $1$ sur sa copie.
$p_B$ ajoute en parallèle l'élément $2$ sur sa copie.
Les copies divergent~: les deux pairs obtiennent des singletons distincts.
Ils synchronisent ultérieurement leurs copies pour échanger et intégrer leurs modifications respectives.
Les modifications ne sont pas intégrées dans le même ordre sur les copies~: ils intègrent la modification de l'autre pair après leur propre modification.
Les copies finissent par converger~: les pairs obtiennent le même ensemble.}\label{fig:optimistic-replication}
\end{figure}


\section{Convergence des copies}\label{sec:convergence}
\index{convergence-des-copies@Convergence des copies}

%Un protocole de réplication optimiste autorise les modifications en concurrence sans coordination.
%La divergence des copies est donc inévitable.

Un protocole de réplication optimiste\index{replication-optimiste@Réplication optimiste de données} peut garantir plusieurs modèles de cohérence dits \enquote{faibles}.
L'une des garanties élémentaires de ces modèle concerne la convergence des copies.
En l'absence de modifications, les copies devraient converger.
Toute interrogation sur deux copies convergentes donne la même réponse.
La convergence peut être définie de différentes manières.
La \emph{cohérence à terme}~\autocite{terry_sessionguarentees_1994,saito_2005_optimisticreplication, vogels_eventuallyconsistent_2009} offre la garantie de convergence la plus faible qu'un protocole de réplication optimiste doit respecter.
Elle garantit que deux copies finissent par converger dès lors qu'elles ont intégré le même ensemble de modifications.
Elle garantit également qu'une modification intégrée par un pair finit par l'être par les autres pairs.

\begin{definition}[Cohérence à terme]\label{def:eventual-consistency}
  Une exécution qui respecte la cohérence à terme, respecte les propriétés suivantes~:
  \begin{description}
  \item[\namedlabel{itm:ec1}{EC1} (intégration à terme)] Une modification intégrée sur la copie d'un pair est à terme intégrée par l'ensemble des pairs.
  \item[\namedlabel{itm:ec2}{EC2} (convergence à terme)] Les pairs qui ont intégré le même ensemble de modifications ont à terme, des copies convergentes.
  \end{description}
\end{definition}

\clearpage

La convergence des copies est généralement une tâche difficile.
Cette difficulté réside principalement dans la présence de conflits de modification.
La \autoref{fig:modif-conflict} présente un scénario où un conflit de modification survient.
La réplication optimiste fait l'hypothèse \textquote{optimiste} que les conflits de modification sont rares et peuvent être résolus ultérieurement.
Les pairs doivent donc gérer les conflits de modification et se mettre d'accord sur un moyen de les résoudre.

\begin{figure}[htb]
\newcommand*\hsep{2.2}
\newcommand*\vsep{-1.4}
\centering
\begin{tikzpicture}
    % Peers
    \node(A) at (0,0) {$p_A$};
    \node (B) at (0,\vsep) {$p_B$};
    % Event streams
    \path (A)
        to +(\hsep,0) node[
            label=above:{$\trm{add}(1)$}
        ] (a2) {$\bullet$}
        to +(2*\hsep,0) node[
            label=above:{$\trm{rmv}(1)$}
        ] (a3) {$\bullet$}
        to +(3*\hsep,0) node[
            label=above:{$\trm{rd}\emptyset$}
        ] (a4) {$\bullet$}
        to +(4*\hsep,0) node (a5) {$\bullet$}
        to +(5*\hsep,0) node (a6) {$\bullet$}
        to +(6*\hsep,0) node[
            label=above:{$\trm{rd}\set*{1}$}
        ] (a7) {$\bullet$}
        to +(7*\hsep,0) node (aend) {}
    ;
    \path (B)
        to +(\hsep,0) node[
            label=below:{$\trm{add}(1)$}
        ] (b2) {$\bullet$}
        to +(3*\hsep,0) node[
            label=below:{$\trm{rd}\set*{1}$}
        ] (b3) {$\bullet$}
        to +(4*\hsep,0) node (b4) {$\bullet$}
        to +(5*\hsep,0) node (b5) {$\bullet$}
        to +(6*\hsep,0) node[
            label=below:{$\trm{rd}\emptyset$}
        ] (b6){$\bullet$}
        to +(7*\hsep,0) node (bend) {}
    ;
    % Timeline
    \foreach \src/\dest in {A/a2,a2/a3,a3/a4,a4/a5,a5/a6,a6/a7,B/b2,b2/b3,b3/b4,b4/b5,b5/b6,a6/aend,b6/bend}
        \draw[timeline] (\src) to (\dest);
    \draw[hb] (b4) to (a6);
    \draw[hb] (a5) to node[fill=white]{\footnotesize sync} (b5);
\end{tikzpicture}
\caption[Conflit de modification d'un ensemble répliqué]{Exemple de conflit de modification.
Les pairs $p_A$ et $p_B$ répliquent un ensemble qui supporte une opération d'ajout $\set*{\trm{add}(n) \given n \in \mathbb{N}_0}$, une opération de suppression $\set*{\trm{rmv}(n) \given n \in \mathbb{N}_0}$, et une opération de lecture $\trm{rd}$.
$p_A$ ajoute l'élément $1$ puis le supprime. En parallèle, $p_B$ ajoute également l'élément $1$.
L'ajout et la suppression en parallèle d'un même élément constitue un conflit de modification.
L'élément ne peut pas être à la fois présent et absent de l'ensemble.
L'intégration \textquote{naïve} des modifications dans des ordres distincts conduit à une divergence des copies.}\label{fig:modif-conflict}
\end{figure}

% coordination -> perf en baisse
% perf en baisse -> moins bonne vivacité ?
% Contraindre plus la convergence pour meilleur vivacité.

% Rappeler que les coordinations sont facteurs de lenteurs ?

Le succès d'une collaboration est conditionné par la capacité des copies à converger.
\textcite{ignat_2015_user-and-delay} ont montré que les activités de collaboration en simultané étaient compromises lorsque le délai entre des modifications de contenu et leurs observations par les autres était trop important.
Dans l'idéal, la convergence devrait être atteinte le plus rapidement possible.
Pour réduire au maximum les délais de convergence, toute coordination devrait être évitée~\autocite{bailis_coordavoidance_2014}.

\begin{figure}[htb]
\newcommand*\hsep{1.7}
\newcommand*\vsep{-1.4}
\centering
\begin{tikzpicture}
    % Peers
    \node(A) at (0,0) {$p_A$};
    \node (B) at (0,\vsep) {$p_B$};
    % Events
    \path (A)
        to +(\hsep,0) node[
            label=above:{$\trm{add}(1)$}
        ] (a2) {$\bullet$}
        to +(2*\hsep,0) node[
            label=above:{$\trm{rmv}(1)$}
        ] (a3) {$\bullet$}

        to +(3*\hsep,0) node (a4) {$\bullet$}
        to +(4*\hsep,0) node (a5) {$\bullet$}
        to +(5*\hsep,0) node[
            label=above:{$\trm{rd}\set*{1}$}
        ] (a6) {$\bullet$}
        to +(8*\hsep,0) node[
            label=above:{$\trm{rd}\set*{1}$}
        ] (a7) {$\bullet$}
        to +(9*\hsep,0) node (aend) {}
    ;
    \path (B)
        to +(\hsep,0) node[
            label=below:{$\trm{add}(1)$}
        ] (b2) {$\bullet$}

        to +(3*\hsep,0) node (b3) {$\bullet$}
        to +(4*\hsep,0) node (b4) {$\bullet$}
        to +(5*\hsep,0) node[
            label=below:{$\trm{rd}\emptyset$}
        ] (b5){$\bullet$}
        to +(8*\hsep,0) node[
            label=below:{$\trm{rd}\set*{1}$}
        ] (b6){$\bullet$}
        to +(9*\hsep,0) node (bend) {}
    ;
    % Coordinations
    \path (A)
        to ++(6*\hsep,1) coordinate (coord1)
        to node[above,midway,
            label=above:{coordination}
        ]{} +(\hsep,0);
    % Timelines
    \foreach \src/\dest in {A/a2,a2/a3,a3/a4,a4/a5,a5/a6,a6/a7,B/b2,b2/b3,b3/b4,b4/b5,b5/b6,a6/aend,b6/bend}
        \draw[timeline] (\src) to (\dest);
    % Coordination
    \draw[fill=white] (coord1) rectangle +(\hsep,-2+\vsep);
    % Sync
    \draw (b3) edge[hb] (a5);
    \draw (a4) edge[hb] node[fill=white]{\footnotesize sync} (b4);
\end{tikzpicture}
\caption[Réplication optimiste et coordination]{Exemple de coordination après l'intégration de modifications.
Les pairs $p_A$ et $p_B$ intègrent les modifications de chacun.
Les copies ne convergent pas.
Les pairs se coordonnent après l'intégration des modifications afin de converger.}\label{fig:convergence-post-coord}
\end{figure}


Un protocole de réplication optimiste garantit une \emph{convergence à terme}\index{convergence-a-termes@Convergence à terme} des copies des pairs.
La convergence à terme assure seulement que les pairs qui ont intégré un même ensemble de modifications convergeront dans un futur plus ou moins proche.
Elle ne donne aucune borne sur le temps nécessaire et le nombre de communications nécessaires  pour converger.
La \autoref{fig:convergence-post-coord} illustre un scénario dans lequel deux pairs intègrent leurs modifications respectives et se coordonnent ensuite afin de converger.

Cette propriété de convergence est faible car il s'agit d'une propriété de vivacité.
Une propriété de vivacité peut être violée en plusieurs points de l'exécution d'un protocole~\autocite{alpern_liveness_1985}.
Elle décrit \enquote{quelque chose de positif} qui devrait survenir.
La formulation d'une propriété de vivacité inclut généralement l'expression \enquote{à terme}.

Afin de faire converger plus rapidement les copies et exclure l'utilisation de coordinations, nous devons contraindre la convergence avec des propriétés de sûreté.
Une propriété de sûreté décrit \enquote{quelque chose de mauvais} qui ne devrait jamais survenir.
Une propriété de sûreté doit être respectée en tout point de l'exécution d'un protocole.
\textcite{shapiro_2011_crdt} proposent la \emph{convergence forte}\index{convergence-forte@Convergence forte} comme propriété de sûreté de la convergence.

\begin{definition}[Convergence forte]~\autocite{shapiro_2011_crdt}\label{def:strong-convergence}
Les pairs qui ont intégré le même ensemble de modifications ont des copies convergentes.
\end{definition}

La \emph{convergence forte} ne nécessite pas de coordination après l'intégration des modifications.
Cependant, elle n'exclut pas l'utilisation de coordination avant l'intégration des modifications.
Les pairs peuvent décider d'un moyen de résoudre les éventuels conflits de modification.
Par exemple, un protocole peut utiliser un pair privilégié qui ordonne les modifications selon un ordre global~\autocite{terry_bayou_1995}.
\textcite{mahajan_2011_cac} proposent la \emph{convergence unidirectionnelle} pour éviter toute coordination.
L'idée centrale de la \emph{convergence unidirectionnelle} est que deux pairs $p_A$ et $p_B$ devraient être capables de faire converger leur copie en deux étapes de communication unidirectionnelle~: $p_A$ envoie des modifications à $p_B$, puis $p_B$ envoie des modifications à $p_A$.

%\begin{definition}[Convergence partielle]\label{def:partial-convergence}
%$p_A$ \emph{converge partiellement} avec $p_B$ si et seulement s'il suffit que
%\begin{enumerate}[label=(\roman*)]
%\item $p_A$ et $p_B$ arrêtent de modifier leur copie et de communiquer avec les autres pairs
%\item $p_B$ transmette à $p_A$ les modifications qu'il a intégré et $p_A$ intègre ces dernières
%\end{enumerate}
%pour que les copies de $p_A$ et $p_B$ convergent.
%\end{definition}
%
%\begin{definition}[Convergence unidirectionelle]\label{def:one-way-convergence}
%Pour tout pair $p_A$ qui arrête de modifier sa copie et de communiquer avec tout pair, il suffit que $p_A$ transmette à $p_B$ les modifications qu'il a intégré et que $p_B$ intègre ces dernières pour que $p_A$ \emph{converge partiellement} avec $p_B$.
%\end{definition}
%
%\vicnote{Y-a t-il un intérêt à renforcer cette propriété ? Non... => supprimer ce qui arrive après ?}
%
%Nous pouvons renforcer cette propriété en exigeant que deux pairs $p_A$ et $p_B$ soient capable de converger en initiant deux communications unidirectionnelles en concurrence~: $p_A$ envoie des modifications à $p_B$, et $p_B$ envoie en parallèle des modifications à $p_A$.
%
%\begin{definition}[Convergence unidirectionelle forte]\label{def:strong-one-way-convergence}
%Si $p_A$ et $p_B$ arrêtent de modifier leur copie et arrêtent de communiquer avec les autres pairs, alors il suffit que
%\begin{enumerate}[label=(\roman*)]
%\item $p_A$ transmette à $p_B$ les modifications qu'il a intégré
%\item $p_B$ transmette (éventuellement en parallèle) à $p_A$ les modifications qu'il a intégré
%\item $p_A$ et $p_B$ intègrent (éventuellement en parallèle) les modifications qu'ils ont reçu de chacun
%\end{enumerate}
%pour que leurs copies convergent.
%\end{definition}

La non-nécessité de coordination pour obtenir des copies convergentes suggère une résolution déterministe des conflits de modification.
Dans le cas de la bibliothèque, nous pouvons supposer que dans la majorité des cas, des emprunts simultanés concernent des livres ou des exemplaires différents sur des périodes distinctes.
Toutefois, des conflits d'emprunts peuvent survenir.
Le nombre d'exemplaires d'un nouveau livre peut être insuffisant par rapport au nombre de lecteur·ice·s intéressé·e·s.
Plusieurs stratégies peuvent être mises en place pour résoudre ces conflits.
On peut considérer que l'emprunteur·ice final·e correspond à la personne qui a fait l'emprunt en premier.

%Au fil des modifications et de leurs échanges, les pairs suivent des cycles de divergence et de convergence.


% \section{Intention des opérations}

% Il peut exister plusieurs manière de réconcilier les copies d'un contenu partagé.
% Dans la \autoref{fig:optimistic-replication}, les pairs intègrent les ajouts de chaque pair afin de converger vers un état équivalent.
% Cependant, rien n'empêche de réconcilier les copies en rejetant les ajouts effectués en parallèle.
% Les pairs $p_A$ et $p_B$ pourraient ainsi converger vers un ensemble vide.

% Cette approche de réconciliation pourrait surprendre les collaborateur·ice·s.
% En effet, si deux éléments sont ajoutés en parallèle dans l'ensemble, on pourrait s'attendre à ce qu'ils appartiennent dans l'ensemble après la réconciliation des copies.
% \textcite{sun_1998_cci} parlent d'intention des opérations.
% Intuitivement, l'intention d'une opération capture l'effet observable de l'opération.
% Cette effet devrait être observable quelque soit le pair qui exécute l'opération.

% L'intégration d'opération concurrentes ou plus largement la réconciliation de copies devrait préserver autant que possible l'intention des opérations qui sont exécutées.
% L'intention d'une opération peut être exprimée comme une postcondition.
% Cette postcondition doit être prudemment spécifiée afin d'assurer son respect quelque soit l'ordre d'exécution des opérations qui sont indépendantes de l'opération considérée.
% Les opérations indépendantes correspondent aux opérations concurrentes et éventuellement à d'autres opérations.

% Reprenons l'exemple de l'ensemble répliqué pour illustrer la spécification de l'intention des opérations.
% Le type abstrait de donnée Ensemble spécifie l'ajout d'un élément dans un ensemble comme l'ensemble auquel a été ajouté l'élément.
% Dans un premier temps nous pouvons définir l'intention de l'ajout d'un élément comme sa spécification.
% Dans la \autoref{fig:optimistic-replication} l'opération $\trm{add}(1)$ a comme intention l'obtention du singleton $\set*{1}$.
% L'opération $\trm{add}(2)$ a comme intention l'obtention du singleton $\set*{2}$.
% La réconciliation des copies des deux pairs conduit nécessairement à la violation de l'intention de l'une des opérations.
% Nous pouvons définir plus prudemment l'intention des opérations de manière à ce qu'elle offre une garantie pertinente tout en évitant leur violation.
% Dans notre cas, nous pouvons définir l'intention d'un ajout comme l'obtention d'un ensemble qui contient à la fois l'élément ajouté et les éléments précédemment ajoutés.
% La présence d'éléments ajoutés en concurrence n'est donc pas exclue.
% Dans la \autoref{fig:optimistic-replication} l'opération $\trm{add}(1)$ a comme intention l'obtention d'un ensemble qui contient l'élément $1$.
% L'opération $\trm{add}(2)$ a comme intention l'obtention d'un ensemble qui contient l'élément $2$.

% % Intention -> No rollbacks
% La préservation de l'intention des opérations peut plus largement retranscrire le principe d'intégrer les opérations de chaque pair sans les ignorer.
% Ce qui empêche la mise en place de retour en arrière.


\section{Modèle du système}

\begin{figure}[htb]
\centering
\includegraphics{fig/impl.pdf}
\caption[Modèle du système]{Une infrastructure de collaboration composée de trois pairs qui répliquent le contenu partagé. Les trois pairs sont connectés par un réseau asynchrone.}\label{fig:collab-system}
\end{figure}

Dans les infrastructures pair-à-pair de collaboration, chaque collaborateur·ice est associé·e à un pair qui lui est exclusif.
\textcite{guerraoui_2016_tradeoffs-replication} parlent de \emph{sticky availability}.
L'une des principales implications est que des pairs peuvent joindre et quitter le système de réplication au gré des connexions et des déconnexions des collaborateur·ice·s.
De nouveaux pairs peuvent joindre lorsque de nouveaux et nouvelles collaborateur·ice·s joignent le groupe
Des pairs existants peuvent se déconnecter indéfiniment lorsque les collaborateur·ice·s associé·e·s quittent le groupe.

Pour interroger et modifier le contenu partagé, un·e collaborateur·ice soumet des opérations au pair qui lui est associé.
Ces opérations dépendent du type du contenu répliqué.
Dans la \autoref{fig:optimistic-replication}, l'ensemble répliqué peut être modifié par une opération d'ajout $\set*{\trm{add}(n) \given n \in \mathbb{N}_0}$ et peut être intérrogé par une opération de lecture $\trm{rd}$ qui donne la composition de l'ensemble.
Les opérations sont d'abord intégrées sur la copie que le pair détient.
Les pairs synchronisent leur copie en arrière-plan.
Une synchronisation peut s'effectuer après chaque nouvelle modification ou peut être espacée dans le temps.

Nous supposons qu'un·e collaborateur·ice ne peut être dissocié·e du pair qui lui est attaché.
La copie est donc \emph{toujours disponible}~\autocite{mahajan_2011_cac} pour être interrogée et modifiée.


\section{Adversaire du système}
\index{advseraire@Adversaire du système}

Un protocole de réplication optimiste doit respecter un ensemble de propriétés de cohérence.
Pour assurer ces propriétés, un protocole doit résister aux défaillances qu'il rencontre au cours de ses exécutions.
Par exemple, il doit se prémunir contre les défaillances inhérentes aux réseaux.

Les protocoles de réplication optimiste supposent généralement l'absence de comportements malintentionnés.
Cette hypothèse n'est pas réaliste au sein d'activités massives de collaboration.
Plus un groupe de collaborateur·ice·s est large, plus il est probable que le groupe inclue un ou plusieurs collaborateur·ice·s malintentionné·e·s.

Les défaillances d'un système peuvent être accidentelles ou résultantes de comportements malintentionnés.
Sans perte de généralité, nous considérons que toute défaillance est le résultat de comportements malintentionnés.
Ces comportements sont exprimés par l'adversaire du système.
L'adversaire du système est mû par un objectif et contrôle des ressources telles que le réseau et des pairs, pour atteindre son objectif.

L'objectif de notre adversaire est de compromettre la convergence des copies des collaborateur·ice·s honnêtes~\footnote{Dans la littérature le terme \textquote{correct} est également utilisé.}.
Et ce dans le but de faire échouer leur collaboration.
Pour simplifier les références ultérieures, nous proposons de classifier les capacités de l'adversaire à travers trois profils~: \emph{réseau asynchrone non-fiable}, \emph{réseau corrompu}, \emph{pairs perturbés}, et \emph{pairs malintentionnés}.

\paragraph{Réseau asynchrone non-fiable.} L'adversaire contrôle le réseau et peut exposer l'ensemble des comportements observés sur un réseau asynchrone non-fiable~\autocite{lynch_1996_asyncnet}.
En particulier, les messages échangés entre les pairs peuvent être réordonnés, dupliqués, retardés, et omis.
L'adversaire peut séparer les pairs en partitions.
Toutefois il ne peut pas empêcher les pairs honnêtes de s'échanger à terme un nombre non-borné de messages.
Ce qui implique qu'il ne peut pas maintenir indéfiniment une partition réseau ou omettre tous les messages envoyés par un pair.
Cette limitation est nécessaire pour que des propriétés de vivacité, telles que l'\emph{intégration à terme}, soient réalisables.

\paragraph{Réseau asynchrone corrompu.} L'adversaire a les limitations et les capacités décrites dans le profil \emph{réseau asynchrone non-fiable}.
En outre, il peut contrefaire des messages.
Il peut par exemple changer l'émetteur d'un message ou le contenu d'un message.

\paragraph{Pairs restaurés.} L'adversaire peut perturber un pair honnête.
Il peut le soumettre à des lenteurs arbitraires.
Il ne peut toutefois pas empêcher un pair de progresser dans l'exécution du protocole.
L'adversaire peut également produire des pannes qui rendent le pair momentanément indisponible.
Nous supposons que les pairs honnêtes disposent d'une mémoire durable et fiable de sorte à ce qu'ils retrouvent l'état dans lequel ils se trouvaient juste avant leur dernière panne.
Pour les mêmes raisons que celles évoquées dans le profil \emph{réseau asynchrone non-fiable}, l'adversaire ne peut pas provoquer des pannes dans l'objectif d'arrêter l'exécution définitive du protocole ou d'empêcher indéfiniment les pairs honnêtes de s'échanger un nombre non-borné de messages.

\paragraph{Pairs malintentionnés.} L'adversaire contrôle un ensemble de pairs malintentionnés~\footnote{Dans la littérature, le terme \textquote{défectueux} est également utilisé.} qui peuvent librement travailler de concert.
Il peut librement connecter de nouveaux pairs malintentionnés et déconnecter, éventuellement indéfiniment, des pairs malintentionnés. Les pairs malintentionnés peuvent se comporter de manière arbitraire.

\paragraph{}L'adversaire peut utiliser l'ensemble de ses capacités et de ses ressources pour exprimer des comportements complexes.
Toutefois, nous supposons les limitations suivantes~:

\begin{itemize}
  \item L'adversaire a des ressources limitées en puissance de calcul et en temps.
  Ces ressources ne sont pas suffisantes pour compromettre les primitives de cryptographies, telles que les signatures digitales et les fonctions de hachage résistantes aux collisions.
  \item Il ne peut pas empêcher un pair honnête d'authentifier la clef publique d'un autre pair honnête.
  L'adversaire ne peut pas obtenir la clef privée d'un pair honnête et il ne peut donc pas usurper son identité cryptographique.
\end{itemize}

Les protocoles de réplication optimiste s'intéressent généralement à un adversaire qui a les profils \emph{réseau asynchrone non-fiable} et \emph{pairs restaurés}.
En ajoutant les deux autres profils \emph{réseau corrompu} et \emph{pairs malintentionnés} nous obtenons un adversaire qui est Byzantin~\autocite{lamport_1982_byzantinegeneralsproblem}\index{adversaire-byzantin@Adversaire Byzantin}.

\clearpage

Certain·e·s auteur·ice·s font la distinction entre les pairs malintentionnés et les pairs honnêtes-mais-défaillants.
Un pair honnête-mais-défaillant expose des comportements qui peuvent être prêtés à un pair malintentionné.
Un pair honnête n'a pas la capacité de décider si un comportement nuisible d'un pair est le résultat d'une défaillance ou de motivations malintentionnées.
Nous considérons donc les pairs honnêtes-mais-défaillants comme des pairs malintentionnés.
Nous supposons qu'il existe au moins un pair honnête au sein de la collaboration.


\section{Synthèse et objectifs}

Un protocole de réplication optimiste autorise la modification et l'interrogation des copies d'un contenu répliqué sans aucune coordination.
Les copies sont donc toujours disponibles pour être modifiées et interrogées.
La modification concurrente des copies conduit inévitablement à leur divergence.
La convergence des copies est une propriété essentielle des infrastructures de collaboration.
Elle conditionne le succès d'une collaboration.
Afin d'assurer des latences faibles et un passage à l'échelle du protocole, nous nous intéressons à des propriétés de convergence qui ne nécessitent pas de coordination.

Le passage à l'échelle et les latences faibles des protocoles de réplication optimiste auxquels nous nous intéressons permettent le support de collaboration massive.
Ce type de collaboration a une probabilité plus importante de rencontrer des comportements malintentionnés.
Nous avons choisi de nous intéresser à des protocoles qui résistent à un adversaire Byzantin.
Ce dernier contrôle le réseau et un ensemble de pairs malintentionnés.
Il peut également perturber le fonctionnement des pairs honnêtes.

Dans ce manuscrit nous cherchons à assurer la convergence des copies des pairs honnêtes sans la nécessité de coordination.
