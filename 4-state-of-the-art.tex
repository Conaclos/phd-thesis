
\chapter{État de l'art}
\label{ch:state-of-the-art}

\minitoc
\bigskip


\section{Journaux authentifiés}

Les journaux authentifiés~\autocite{mahajan_depot_2011,truong_authenticating_2012} sont des journaux qui ne peuvent pas être falsifiés.
L'adversaire ne peut pas omettre, dupliquer, ou réordonner des opérations sans qu'un pair honnête s'en aperçoive.
Pour ce faire, les opérations sont signées par le pair qui les émet et leur identifiant contient une empreinte cryptographique (\emph{hash}) qui résiste aux collisions.


\subsection{Exemple}

Suppose qu'une équipe de volontaires co-organisent un événement.
Alice créait un contenu partagé pour faciliter l'organsinions de l'événement.
Elle invite plusieurs organisateurs à prendre part à la collaboration.
Chaque année, des perturbateurs tentent de compromettre le bon déroulement de l'évènement.
Puisque l'équipe de volontaires est ouverte à tous et toutes, des volontaires mal-intentionné·e·s, tel que Mallory, rejoignent le groupe.
Afin de protéger la convergence du contenu partagé, un journal \ac{VFJC} est utilisé.

Alice, Mallory, et d'autres membres participent dans plusieurs sessions de collaboration.
Le contenu paratgé contient des rapports, des compte-rendu de réunion, et un sondage pour recenser les personnes disponible le 4 mai., date de l'événement.
Les collaborateur·ice·s obtiennent des journaux convergents.

%La~\autoref{fig:collaborative-example} la partie de la collaboration qui nous intéresse.
Alice ($P_A$) invite Bob ($P_B$) et demande à ses collaborateurs de remplir le sondage.
$c^A_1$ somme ces actions.
Bob rejoint le gtroupe grâce à Mallory ($P_M$).
Mallory tente de tromper Bob en modifiant la date de l'événement.
Elle modifie donc la date au 8 Mars et partage cette modification $c^M_2$ uniquement avec Bob.
Mallory génère également $c^M_1$ où elle confirme à Alice sa présence pour le 4 Mai.
Bob génère $c^B_1$ où il confirme sa présence pour le 8 Mars.
A la réception de $c^B_1$, Alice est consciente que Bob n'a pas le même contexte causal qu'elle.
Prenant $c^M_2$, elle détecte l'équivocation de Mallory.
A la fin de la session illustrée, les journaux de Alice et Bob convergent.
Alice et Bob arrêtent de communiquer avec Mallory.

%\begin{figure}[ht]
%  \begin{subfigure}{\linewidth}
%    \centering
%    \begin{utikzhbgraph}
%      \node (A) at (0,0) {$P_A$};
%      \node (M) at (0,-2) {$P_M$};
%      \node (B) at (1,-4) {$P_B$};
%      \draw[timeline] (A) to (8,0);
%      \draw[timeline] (M) to (8,-2);
%      \draw[timeline] (B) to (8,-4);
%      \uevent{a1}{1, 0}{$c^A_1$}
%      \uevent{m1}{2.4,-1.55}{$c^M_1$}
%      \uevent{m2}{2.4,-2.45}{$c^M_2$}
%      \uevent{b1}{3.8,-4}{$c^B_1$}
%      \uevent{a2}{5.2,0}{$c^A_2$}
%      \uevent{b2}{6.8,-4}{$c^B_2$}
%      \draw[com] (a1) to (1.6,-2);
%      \draw[com] (m1) to (3,0);
%      \draw[com] (m2) to (3,-4);
%      \draw[com] (b1) to (4,-2);
%      \draw[com] (b1) to (4.8,0);
%      \draw[com] (a2) to (5.6,-4);
%      \draw[com] (b2) to (7.4,0);
%    \end{utikzhbgraph}
%    \caption{Dashed arrows are transmissions of one or more contributions.}\label{fig:communication-graph}
%  \end{subfigure}
%  \begin{subfigure}{\linewidth}
%    \centering
%    \begin{utikzhbgraph}
%      \uevent{a1}{0, 0}{$c^A_1$}
%      \uevent{m1}{2, 1}{$c^M_1$}
%      \uevent{a2}{4, 1}{$c^A_2$}
%      \uevent{m2}{2, -1}{$c^M_2$}
%      \uevent{b1}{4, -1}{$c^B_1$}
%      \uevent{b2}{6, 0}{$c^B_2$}
%      \uhb{a1}{m1}
%      \uhb{m1}{a2}
%      \uhb{a1}{m2}
%      \uhb{m2}{b1}
%      \uhb{a2}{b2}
%      \uhb{b1}{b2}
%    \end{utikzhbgraph}
%    \caption{A \ac{VFJC} log with a fork junction at $c^B_2$.}\label{fig:vjc-log-fork-junction}
%  \end{subfigure}
%
%  \caption{\subref{fig:communication-graph} A collaborative session and \subref{fig:vjc-log-fork-junction} the \ac{VFJC} log of honest participants (Alice $P_A$ and Bob $P_B$) after their convergence. Mallory ($P_M$) issues two non-linear contributions ($c^M_1 \| c^M_2$). She is provably malicious.}\label{fig:collaborative-example}
%\end{figure}
