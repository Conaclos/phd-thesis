
\chapter{Conclusion}\label{ch:conclusion}

\minitoc{}
\bigskip

% Ce dernier chapitre montre comment nous avons répondu aux questions de recherches que nous nous sommes posées et les limites auxquelles nous nous sommes heurtées.
% Il propose plusieurs axes de recherche à explorer.

Les infrastructures pair-à-pair de collaboration et la réplication optimiste des contenus partagés ouvrent la voie vers la conception d'applications de collaboration hautement disponibles, aux latences faibles, qui tolèrent les partitions réseaux, et qui passent à l'échelle.
Les protocoles de réplication optimiste proposés pour l'édition collaborative d'un contenu sont généralement vulnérables aux attaques d'un adversaire actif et ne tiennent pas toujours leurs promesses dans des cas d'utilisation exigeants tels que l'édition collaborative de texte en simultané.

Dans ce manuscrit, nous avons proposé deux protocoles qui permettent de répliquer de manière sûre un contenu, et un protocole de réplication optimiste de séquence qui résiste mieux aux aléas du réseau.
Ce dernier chapitre est pour nous l'occasion de rappeler nos questions de recherche et d'y répondre.
Il synthétise nos contributions et met en évidence leurs limites.
Nous proposons plusieurs axes de recherche pour poursuivre nos travaux.


\clearpage

% %%%
% Tu reprends tes objectifs du départ et tu dis comment
% tu as fait avancer l'état de l'art
% %%%
% pointer les limites de tes contributions dans la 
% conclusion plutôt qu'à la fin de chaque chapitre
% %%%

\section{Convergence sécurisée à l'aide d'un journal tronqué}

\subsection{Résumé des contributions}

La communauté scientifique a proposé de nombreux protocoles de réplication optimiste dans le but de concevoir des applications de collaboration hautement disponibles, aux latences faibles, et qui passent à l'échelle.
La plupart des protocoles de réplication optimiste ne garantissent pas la convergence des copies du contenu en présence de pairs malintentionnés.
Les pairs malintentionnés peuvent en effet produire des équivoques pour compromettre la convergence des copies des pairs honnêtes.
Nous nous sommes donc demandés s'il était possible de protéger la convergence des copies des pairs honnêtes tout en préservant les propriétés désirées des protocoles de réplication optimiste.

Nous avons proposé deux protocoles qui protègent la convergence des copies des pairs honnêtes.
Notre premier protocole repose sur le maintien d'un journal infalsifiable et répliqué.
Chaque pair maintient son propre journal dans lequel il enregistre les messages qu'il transmet aux autres pairs et les messages qu'il reçoit des autres pairs.
Notre protocole requiert la préservation de l'intégralité du journal pour maintenir la cohérence des copies et permettre à de nouveaux pairs de joindre la collaboration.
La préservation de l'intégralité du journal et sa transmission aux nouveaux pairs engendrent des coûts en mémoire et en communication réseau.

Pour remédier à ces problèmes, nous avons proposé un second protocole qui repose sur le premier et permet la troncature du journal.
La troncature du journal est effectuée de manière à protéger la cohérence des copies du contenu partagé.
Elle repose sur le concept de \emph{stabilité}.
Un message du journal est stable dès lors qu'il devient une dépendance de tout message ajouté ultérieurement dans le journal.
Nous avons proposé un algorithme qui permet de déterminer si un message du journal est stable.
Cet algorithme prend en compte l'invitation de nouveaux pairs et l'éviction de pairs malintentionnés.
Il a une complexité linéaire avec le nombre de messages dans le journal.
Pour constituer sa copie du contenu partagé, un nouveau pair récupère un journal tronqué et l'état de la copie associée au journal.
Il authentifie l'état de la copie à l'aide du journal tronqué.

Pour montrer la validité de nos protocoles, nous avons proposé un nouveau modèle de cohérence que nous avons nommé \acl{VFJC} Dynamique.
Il s'agit d'une restriction du modèle de cohérence \acf{VFJC}.
Son introduction nous a également amené à proposer le modèle de cohérence Causale Dynamique.
Les deux modèles de cohérence que nous avons introduits tiennent compte de la dynamique des groupes et permettent la définition du concept de stabilité.
Nous avons également défini le concept de stabilité pour le modèle de cohérence \acs{VFJC} et le modèle de cohérence Causale.

Nos deux protocoles ne nécessitent pas de coordination entre les pairs.
Chaque pair peut à tout moment modifier le contenu répliqué ou tronquer son journal.
Par ailleurs, nos protocoles prennent en compte la dynamique des groupes collaboratifs et évincent les pairs qui sont reconnus malintentionnés.

\subsection{Limites et perspectives}

% %%%
% Tu organises tes perspectives en fonction de leur
% importance et terme
% Les choses à court terme puis à moyen terme et à
% plus long terme et surtout tu rédiges. Tu ne fais
% pas juste des bullet point
% %%%
% Tu mets tout au même niveau, des trucs un peu basiques
% (éviction des pairs inactifs) et des trucs avec des 
% perspectives plus grandes (consensus asynchrone)
% Et tu ne classes rien
% il y a des problématiques futures pour 
% - La sûreté
% - La sécurité
% - L'utilisabilité
% - Les performances
% %%%

Nous avons apporté des éléments de preuves pour valider nos protocoles.
Ces preuves devraient être complétées.
Il reste à démontrer que l'exécution de notre protocole à journaux tronqués respecte le modèle de cohérence \acl{VFJC} Dynamique.
Nous devrions mener des expérimentations pour évaluer leur capacité à passer à l'échelle.

Pour réduire la complexité algorithmique de la troncature du journal, nous n'identifions pas l'ensemble des messages qui peuvent être supprimés.
Nous effectuons une approximation qui nous permet de supprimer un sous-ensemble de ces messages.
Nous devrions évaluer par expérimentation si cette approximation n'est pas trop restrictive.

À plus long terme, nous souhaitons explorer d'autres applications du concept de stabilité et améliorer l'utilisabilité de nos protocoles.
Nous donnons un aperçu de ces axes de recherche dans les paragraphes suivants.

Nous avons développé le concept de stabilité à travers plusieurs modèles de cohérence et le concept de \emph{stabilité convergente} qui garantit que deux pairs avec le même journal ont un même ensemble de messages stables.
Nous avons employé ce concept pour tronquer le journal.
Nous pensons que le concept de stabilité peut trouver de nombreuses applications.
Il pourrait par exemple être le socle d'un protocole de consensus asynchrones~\autocite{bracha1985asynchronous}.
Dans un protocole de consensus asynchrones, un ensemble de pairs s'accordent sur une valeur unique sans aucune coordination en présence de pairs malintentionnés.
Le protocole doit résister aux équivoques des pairs malintentionnés.
Les pairs échangent des messages pour trouver un consensus.
Un message convergent-stable garantit à un pair que tout sous-ensemble de pairs a connaissance du message et qu'il ne peut prétendre en avoir connaissance à un autre sous-ensemble de pairs.
Les pairs peuvent ainsi trouver un consensus de manière déterministe à l'aide de l'ensemble de messages convergents-stables.

Le plus grand problème d'utilisabilité de notre protocole à journaux tronqués réside dans la possibilité de bloquer la troncature du journal.
La troncature du journal repose sur le concept de stabilité.
La stabilisation de messages exige la participation de l'ensemble des pairs qui ont été invités, à l'exception des pairs évincés.
Un pair qui n'exécute pas d'opérations ou qui produit délibérément des messages qui dépendent uniquement de ses anciens messages bloque alors la stabilisation de messages.
Pour remédier à ce problème, nous pourrions évincer les pairs qui empêchent la stabilisation de messages.

Les pairs malintentionnés peuvent générer des messages non-linéaires et ainsi produire des embranchements.
Nos protocoles rejettent les messages non-linéaires (et par extension les embranchements) qui ne sont pas acceptés par au moins un pair présumé honnête et connu.
Les pairs malintentionnés peuvent inviter de nouveaux pairs dans un embranchement.
Les pairs honnêtes peuvent donc être amenés à rejeter des invitations.
Un nouveau pair peut détecter qu'il a été invité dans un embranchement rejeté par les autres pairs.
Il pourrait alors demander à être de nouveau invité par un autre pair.

Deux invitations ne peuvent pas inviter le même pair.
Nos protocoles garantissent cette contrainte en générant l'identifiant du pair à partir du message d'invitation.
Deux messages d'invitation distincts donnent lieu à deux identifiants de pair distincts.
Cette contrainte peut rendre plus difficile l'implémentation des protocoles.
Nous avons introduit cette contrainte pour simplifier l'expression de la stabilité.
Nous souhaitons étudier les conséquences de la suppression de cette contrainte sur l'expression de la stabilité.


\section{Séquence répliquée synchronisée par différences}

\subsection{Résumé des contributions}

De nombreux protocoles de réplication optimiste de séquences de caractères ont été proposés pour supporter l'édition collaborative de texte en simultané.
Pour synchroniser leurs copies du document textuel, les pairs échangent leurs modifications sous la forme d'opérations d'insertion et de suppression.
Les opérations doivent être intégrées une seule fois et dans un ordre spécifique.
Un ordre d'intégration causal des opérations est généralement supposé.

Les aléas du réseau conduisent au réordonnement et à la perte d'opérations.
L'intégration d'opérations peut donc être retardée si l'une de leurs dépendances n'a pas été préalablement intégrée.
Ce qui propage des ralentissements dans l'ensemble de la collaboration.
Par ailleurs, l'échange et l'intégration des opérations après une longue période de déconnexion d'un·e des collaborateur·ice·s peuvent être coûteuses.
Nous nous sommes donc demandé si un protocole de réplication optimiste pour l'édition collaborative de texte en simultané pouvait remédier à ces deux problèmes.

Dans ce manuscrit, nous nous sommes intéressés aux types de données répliquées, communément nommés \acp{CRDT}.
Nous avons proposé et implémenté un nouveau \acs{CRDT} séquence nommé \emph{Dotted LogootSplit}.
Contrairement aux autres \acs{CRDT} séquences, il ne retarde pas l'intégration des modifications des pairs.
Chaque modification est intégrée dès sa réception.
Ce qui évite la propagation de ralentissements, voire l'interruption temporaire, de la collaboration.
Il permet également à deux pairs d'échanger directement leur état.
Ce qui évite des échanges coûteux de nombreuses modifications lorsqu'un pair se reconnecte après une longue période de déconnexion.

Pour ce faire, \emph{Dotted LogootSplit} synchronise les copies par différences d'états.
Une différence d'états peut résumer une seule modification ou plusieurs modifications.
Les différences d'états peuvent être librement réordonnées et dupliquées.
La synchronisation par différences d'états offre une grande flexibilité dans la conception de protocoles de synchronisation.
Notre approche est suffisamment générique pour être adaptée à une famille de \acp{CRDT} séquences que nous avons formalisée.

\emph{Dotted LogootSplit} tire avantage des dernières avancées de l'état de l'art.
Il agrége les méta-données associées aux caractères sous la forme de blocs.
Ce qui le rend particulièrement adapté à l'édition collaborative de texte.

\subsection{Limites et perspectives}

\emph{Dotted LogootSplit} a été implémenté et testé dans des conditions réelles.
Toutefois, nous n'avons pas conduit des expérimentations pour valider son adéquation à l'édition collaborative de texte en simultané avec un nombre important de collaborateur.ice.s.
Pour montrer qu'il se comporte mieux que les protocoles existants face aux aléas du réseau, nous devrions simuler des pertes de messages sur le réseau, des retards de messages, des réordonnements, et des déconnexions temporaires de pairs.
Dans ces conditions, nous devrions mesurer les délais entre le moment où une modification est effectuée par un pair et le moment où elle est intégrée par les autres pairs.

À plus long terme, nous souhaitons \begin{inlinelist}\item améliorer la résistance d'une séquence \emph{Dotted LogootSplit} aux attaques d'un adversaire actif, \item proposer une stratégie de synchronisation adaptative, et \item proposer une stratégie de génération de positions densément ordonnées qui limite les entrelacements de caractères\end{inlinelist}.
Nous donnons un aperçu de ces deux axes de recherche dans les paragraphes suivants.

Les protocoles de réplication optimiste sont généralement conçus en supposant l'absence de pairs malintentionnés.
Les pairs malintentionnés peuvent produire des équivoques en vue de compromettre la convergence des pairs honnêtes.
Certains \acp{CRDT} résistent par conception aux équivoques de pairs malintentionnés.
Les séquences répliquées associent un identifiant unique à chaque valeur.
La synchronisation des copies se base sur ses identifiants uniques.
Un pair malintentionné peut produire une équivoque en associant un même identifiant à différentes valeurs.
Dans le cas des séquences répliquées à positions densément ordonnées, telle qu'une séquence \emph{Dotted LogootSplit}, un pair malintentionné peut également produire une équivoque en associant à une valeur et son identifiant une position distincte.
L'utilisation de nos protocoles à journaux infalsifiables permet de détecter les équivoques.
Nous pourrions proposer des mécanismes pour résoudre les conflits de modification introduits par les équivoques.
Cependant, nous pensons que ces problématiques devraient être prises en compte dans la conception des \acp{CRDT}.
Elles devraient en particulier être prises en compte dans la conception de \acp{CRDT} synchronisés par différences d'états puisque l'utilisation d'un journal infalsifiable annule la plupart de leurs avantages.
En effet, l'utilisation d'un journal infalsifiable nécessite la réintroduction de dépendances causales entre les modifications.

\emph{Dotted LogootSplit} offre une grande flexibilité dans la conception du protocole de synchronisation.
Il peut utiliser un protocole de synchronisation par opérations, un protocole de synchronisation par états complets, ou l'un des protocoles de synchronisation par différences d'états proposés pour les \acp{CRDT}.
Nous pourrions évaluer les performances de différents protocoles de synchronisation en fonction des paramètres que sont la taille du contenu, le nombre de modifications effectuées depuis la dernière synchronisation, le mode de collaboration utilisé (en simultané ou en différé), la qualité du réseau.
À partir de cette évaluation, nous pourrions proposer une stratégie adaptative qui choisit le protocole de synchronisation de manière dynamique au cours de la collaboration.

La modification parallèle d'un contenu partagé rend difficile la définition de l'intention des opérations que les collaborateur·ice·s exécutent.
Concernant l'édition collaborative de texte, il est communément accepté qu'une insertion à un indice donné traduit l'intention de vouloir insérer un caractère entre les caractères qui entourent cet indice.
\textcite{kleppmann2019_interleaving} ont mis en évidence l'importance de rejeter les entrelacements de caractères insérés en parallèle.
\emph{Dotted Logoot} peut produire des entrelacements.
Toutefois, il pourrait tirer avantage de l'agrégeabilité de ses positions pour rejeter certains entrelacements de caractères.
Lorsqu'un pair insère des caractères à un même emplacement les uns à la suite des autres, les caractères sont associés à des positions agrégeables.
Les caractères sont groupés au sein d'un bloc.
Le générateur de position pourrait garantir que tout caractère inséré en parallèle de ces caractères est placé avant ou après ces derniers.


\section{Notes finales}

Les infrastructures pair-à-pair de collaboration et la réplication optimiste des contenus partagés offrent des défis difficiles et stimulants.
Ils nous poussent à remettre en question des pratiques établies et à trouver de nouveaux compromis.
La suppression du serveur et la promotion des collaborateur·ice·s en pairs responsables de la réplication du contenu partagé permettent de dépasser les limites des architectures centralisées, mais apportent aussi de nouvelles problématiques.

Les infrastructures pair-à-pair de collaboration présentent une surface d'attaque plus large qu'une architecture centralisée, étant donnée la responsabilité conférée aux pairs.
Les protocoles de réplication optimiste répondent déjà à de nombreuses problématiques difficiles.
Leur sécurisation est une problématique qui apporte ses propres difficultés.
Nous comprenons aujourd'hui davantage les problèmes que soulève la conception de protocoles de réplication optimiste et la manière d'y répondre.
Les protocoles de réplication optimiste ont fait leurs preuves et nous pensons qu'il est désormais temps de prendre en compte les problématiques de sécurité dès leur conception.

% Certains types d'activités collaboratives mettent encore à rudes épreuves les protocoles de réplication optimiste et mettent en exergue les limites des protocoles de réplication optimiste.
% La conception d'un protocole de réplication de séquence adapté à la co-édition de texte en simultané est particulièrement difficile.
