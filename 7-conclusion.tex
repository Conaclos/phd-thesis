
\chapter{Conclusion}\label{ch:conclusion}

\minitoc{}
\bigskip

Ce dernier chapitre synthétise nos contributions et propose une réponse aux questions de recherches que nous nous sommes posées.
Il propose également des problématiques de recherches pour des travaux ultérieurs.

\clearpage
\section{Résumé des contributions}

\subsection{Convergence sécurisée à l'aide d'un journal tronqué}

La communauté scientifique a présenté de nombreux protocoles de réplication optimiste dans le but de concevoir des applications de collaboration hautement disponible, aux latences faibles, et qui passent à l'échelle.
La plupart des protocoles présentés ne résistent pas à la présence de pairs mal-intentionnés dans la collaboration.
Les pairs mal-intentionnés peuvent produire des équivoques dans le but de compromettre la convergence des copies des pairs honnêtes.

Les pairs peuvent détecter ces équivoques et converger de manière sûr à l'aide de journaux infalsifiables.
Malheureusement l'utilisation de journaux infalsifiables limite le passage à l'échelle de la collaboration.
L'ensemble du journal doit être conservé pour maintenir sa cohérence et permettre à de nouveaux pairs de joindre la collaboration.

Pour remédier à cette limitation nous avons proposé un protocole qui permet la troncature du journal et un protocole qui permet d'authentifier un état du contenu partagé à l'aide d'un journal tronqué.
La troncature du journal repose sur le concept de stabilité.
Nous avons définit le concept de stabilité au sein de plusieurs modèle de cohérence, et en particulier sur deux modèles de cohérence que nous avons introduit~: le modèle de cohérence causale dynamique et le modèle de cohérence \acl{VFJC} dynamique.

Nous avons apporté des éléments de preuves de nos protocoles.
Ces preuves devraient être complétées et une évaluation expérimentale devraient comparer notre protocole à journaux tronqués à un protocole à journaux complets.


\subsection{Séquence répliquée synchronisée par différences}

De nombreux protocoles de réplication optimiste de séquence de caractère sont été proposés pour supporter l'édition collaborative de texte.
Chacun à sa manière a permis d'améliorer l'état de l'art de l'édition collaborative.
Nous avons proposé un nouveau protocole de réplication optimiste de séquence nommé \emph{Dotted LogootSplit}.
Contrairement aux autres protocoles, il ne retarde pas l'intégration des modifications des pairs.
Chaque modification est intégrée dès sa réception.
Ce qui évite la propagation de ralentissement, voire l'interruption temporaire, de la collaboration.
Il permet également à deux pairs d'échanger directement leurs états.
Ce qui évite des échanges coûteux de journaux de modifications lorsque les deux pairs se reconnectent après une longue période de déconnexion.

Pour ce faire, \emph{Dotted LogootSplit} synchronise les copies par différences d'états.
Une différence d'état peut synthétiser une seule modification ou plusieurs modifications.
Les différences d'états peuvent être librement réordonnées et dupliqués.
La synchronisation par différences d'états offre une grande flexibilité dans al conception de protocole de synchronisation.
Des évaluations devraient être menés pour déterminer qu'elle est la meilleure stratégie de synchronisation en fonction des modes de collaboration.


\section{Perspectives}

\paragraph{Applications du concept de Stabilité.} Le concept de stabilité que nous avons développé a été uniquement appliqué pour la troncature sûr de journaux infalsifiables.
Des travaux ultérieurs pourraient s'intéresser à l'application de ce concept à d'autres domaines.
Il pourrait par exemple être le socle pour la conception de protocole de consensus asynchrones~\autocite{bracha1985asynchronous}.

\paragraph{Éviction des pairs inactifs.}
Nous proposons un protocole pour tronquer un journal infalsifiable.
Le mécanisme de troncature se base sur la stabilité des messages du journal.
La déconnexion prolongée d'un pair retarde la troncature du journal.
Elle peut même l'empêcher si la déconnexion est permanente.
Les pairs déconnectés depuis un certain temps pourraient être automatiquement évincés.

\paragraph{Convergence sécurisée des copies du contenu.} Les journaux infalsifiables permettent de détecter les équivoques et ainsi d'obtenir l'ensemble des opérations qui ont été intégrées sur le contenu partagé.
Certains type de données répliquées sont sensibles aux équivoques.
En présence d'équivoque les copies du contenu partagé peuvent diverger de manière permanente.
Par exemple, les protocoles de séquences répliquées sont sensibles aux équivoque qui produisent deux insertions de valeurs distinctes avec le même identifiant.
Les pairs pourraient donc se retrouver divisés entre ceux qui ont la première valeur et ceux qui ont la seconde valeur.

\paragraph{Annulation de modifications.} L'une de nos publication~\autocite{2019_yu_genericundo} propose un mécanisme générique d'annulation et de rejeu de modification au sein de type de données répliquée synchronisés par différences d'états.
Ce mécanisme pourrait être intégré à \emph{Dotted LogootSplit} et pourrait être une version de \emph{Logoot} adapté pour l'annulation et le rejeu de modification~\autocite{weiss2010logoot}.

\paragraph{Entrelacements.} La modification parallèle d'un contenu partagé rend difficile la définition de l'intention des opérations que les collaborateur·ice·s exécutent.
Concernant l'édition collaborative de texte, il est communément accepté qu'une insertion à un indice donné traduit l'intention de vouloir insérer un caractère entre les caractères qui entourent cet indice.
\textcite{kleppmann2019_interleaving} a mis en évidence l'importance de rejeter les entrelacements de caractères insérées en parallèle.
\emph{Logoot}, \emph{LogootSplit}, et notre protocole \emph{Dotted Logoot} peuvent produire des entrelacements.
La génération de positions devraient être soigneusement effectuée pour limiter les entrelacements voire les éviter complètement.

\paragraph{Génération adaptative des positions densément ordonnées.} \emph{LSeq}~\autocite{nedelec_2013_lseq} propose une stratégie de génération des positions sensément ordonnées qui en moyenne réduit leur occupation mémoire.
Cette stratégie a été conçue pour générer des positions qui ne sont pas agréables.
L'adaptation de cette stratégie à des positions agréables peut représenter un défi stimulant.

\paragraph{Protocole de synchronisation par différences d'états.} \emph{Dotted LogootSplit} est le premier protocole de réplication de séquence synchronisée par différences d'états.
La synchronisation par différences d'états offre une grande flexibilité dans la conception de protocole de synchronisation.
Les meilleures stratégies de synchronisation pour l'édition collaborative de texte reste à déterminer.
