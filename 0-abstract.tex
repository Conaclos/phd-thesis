\begin{ThesisAbstract}
\begin{FrenchAbstract}
Une application de collaboration permet à plusieurs individus de coéditer un contenu.
Les infrastructures pair-à-pair de collaboration visent à la conception d’applications de collaboration hautement disponibles, aux latences faibles, qui tolèrent les partitions réseaux, et passent à l’échelle.
Chaque pair (individu) a une copie du contenu qu’il peut indépendamment modifier.
La modification concurrente des copies conduit à leur divergence.
Les protocoles de réplication sont responsables de la convergence des copies du contenu.

Ces protocoles supposent l'absence de pairs mal-intentionnés qui peuvent compromettre la convergence des copies.
Pouvons-nous protéger la convergence des copies et préserver les propriétés des infrastructures pair-à-pair~?
Nous proposons deux protocoles qui protègent la convergence des copies.
Le premier protocole maintient un journal répliqué et infalsifiable qui enregistre les modifications du contenu.
Les pairs conservent l’intégralité du journal pour déjouer les attaques des pairs mal-intentionnés et pour le transmettre à ceux qui rejoignent la collaboration.
Le second protocole permet aux pairs de tronquer leur journal.
La troncature du journal repose sur le concept de \emph{Stabilité}.
Une modification devient stable lorsque toute modification intégrée ultérieurement dans le journal dépend d'elle.
Pour rejoindre la collaboration, un pair récupère une copie et un journal tronqué.
Il vérifie l'authenticité de la copie à partir du journal tronqué.

Un protocole de réplication peut être encapsulé dans un type de données répliquées communément nommé \emph{CRDT}.
Les \emph{CRDTs} séquences de la littérature supposent généralement un ordre d'intégration causal des modifications du contenu.
Le retard d'une modification propage des ralentissements dans l'ensemble du système.
Pouvons-nous éliminer ces ralentissements~?
Nous proposons un \emph{CRDT} séquence qui n'impose pas d'ordre d'intégration des modifications du contenu.
Il résiste mieux aux aléas du réseau.

\KeyWords{adversaire actif, cohérence de données, collaboration, convergence, journaux infalsifiables, pair-à-pair, réplication optimiste, sécurité, CRDT}
\end{FrenchAbstract}

%\begin{EnglishAbstract}
%\end{EnglishAbstract}
\end{ThesisAbstract}